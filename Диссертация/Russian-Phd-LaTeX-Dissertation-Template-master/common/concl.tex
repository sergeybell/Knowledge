%% Согласно ГОСТ Р 7.0.11-2011:
%% 5.3.3 В заключении диссертации излагают итоги выполненного исследования, рекомендации, перспективы дальнейшей разработки темы.
%% 9.2.3 В заключении автореферата диссертации излагают итоги данного исследования, рекомендации и перспективы дальнейшей разработки темы.
\begin{enumerate}
  \item На основе анализа внутрипучкового рассеяния, а также стохастического охлаждения показано, что использование метода 'резонансной' структуры способно увеличить эффективность стохастического охлаждения. Особенно эффективным может быть использование 'комбинированной' структуры. Однако, эффекты ВПР для приведенных структуры оказались в несколько раз большими и в конечном счёте недостаточными, делая предпочтительной 'регулярную' структуры.
  \item Для 'резонансной' структуры может быть варьирована критическая энергия, что использовано для коллайдерных экспериментов с протонами.
  \item Численные исследования показали, что прохождение критической энергии может вызывать нестабильность продольного фазового движения. Использование процедуры скачка способно. 
  \item Экспериментальные данные процедуры скачка критической с синхротрона У-70.
  \item Использование процедуры скачка для коллайдера NICA ограничено величиной скачка критической энергии, а также для гармонического ВЧ темпом изменения критической энергии по сравнению с темпом ускорения пучка. Что делает невозможным использование процедуры для этого типа ВЧ. Для барьерного ВЧ приведены оценки продольной микроволновой неустойчивости, показывающие существенное ограничение на параметры конечного сгустка.
  \item Для исследования спиновой динамики и реализации "квази-замороженного" спина в коллайдере NICA рассмотрено введение обводных каналов bypass.
  \item Модернизированная структура синхротрона Nuclotron 
\end{enumerate}
